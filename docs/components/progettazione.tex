\section{Fase di progettazione}
\label{sec:fase_di_progettazione}
\subsection{Struttura}
%Parliamo dello schema che è ambiguo, perchè ci sono film che sono di più generi, e persone che sono sia attori che registi. La struttura è ibrida, principalmente gerarchica, con elementi di ipertesto. L'utente può visitare la parte informativa del sito utilizzando esclusivamente la parte gerarchica, mentre la parte di ipertesto è indispensabile solo per la parte interattiva. Inoltre, permette di seguire degli shortcut, comodi ma non necessari, fra pagine su rami diversi della gerarchia.
\label{sub:struttura}
\subsubsection{Header}
\label{ssub:header}
% L'header contiene il logo del sito e due link per il login. Si è scelto di non far rimandare il logo a nessuna pagina in quanto il link per la homepage è già contenuto nel menu sotto la voce Home. I due pulsanti per il login e il signup sono rispettivamente "Accedi" e "Registrati" e saranno entrambi presentati solo nel caso in cui l'utente non abbia ancora effettuato l'accesso.
% subsubsection header (end)
\subsubsection{Nav}
\label{ssub:nav}
% Il nav del sito contiene il menù e la barra di ricerca. Il menu è costruito come un \htmlcode{ul}, e contiene link alle pagine principali del sito: la home page, la pagina relativa ai film, la pagina relativa alle serie televisive, e la pagina relativa ad attori e registi. La barra di ricerca è data da un \htmlcode{form} contenente un \htmlcode{input} con \htmlcode{type="text"}; si è scelto di non utilizzare \htmlcode{type="search"} per evitare la necessità di utilizzare lo standard HTML5, e quindi mantenere la compatibilità con più browser possibili.
% subsubsection nav (end)
\subsubsection{Breadcrumb}
\label{ssub:breadcrumb}
% Il breadcrumb è presente in tutte le pagine "ordinarie" del sito, mentre è assente dalle pagine di login e signup,  nel profilo degli utenti registrati, e nelle pagine di inserimento di nuovi contenuti. Infatti, mentre le altre pagine possono essere raggiunte seguendo un percorso concettuale molto regolare a partire dalla home page, le pagine relative agli utenti registrati possono essere raggiunte da qualsiasi pagina nel sito, rendendo impossibile individuare una pagina che sia la "precedente".
% subsubsection breadcrumb (end)
\subsubsection{Content}
\label{ssub:content}

% subsubsection content (end)
\subsubsection{Footer}
\label{ssub:footer}
% Il footer contiene i contatti e i crediti relativi agli autori del sito, e le icone di certificazione della validità rispetto gli standard W3C di HTML (nelle forme XHTML o HTML5 in base alla pagina) e CSS2.
% subsubsection footer (end)
% subsection struttura (end)
\subsection{Visualizzazione} %le pagine devono essere accessibili indipendentemente dalle dimensioni del dispositivo e del browser
\label{sub:visualizzazione}
\subsubsection{Convenzioni interne}
\label{ssub:convenzioni_interne}

% subsubsection convenzioni_interne (end)
\subsubsection{Desktop}
\label{ssub:desktop}

% subsubsection desktop (end)
\subsubsection{Mobile}
\label{ssub:mobile}

% subsubsection mobile (end)
\subsubsection{Stampa}
\label{ssub:stampa}

% subsubsection stampa (end)
% subsection visualizzazione (end)
% section fase_di_progettazione (end)