\section{Fase di progettazione}
\label{sec:fase_di_progettazione}
\subsection{Struttura}
%Parliamo dello schema che è ambiguo, perchè ci sono film che sono di più generi, e persone che sono sia attori che registi. La struttura è ibrida, principalmente gerarchica, con elementi di ipertesto. L'utente può visitare la parte informativa del sito utilizzando esclusivamente la parte gerarchica, mentre la parte di ipertesto è indispensabile solo per la parte interattiva. Inoltre, permette di seguire degli shortcut, comodi ma non necessari, fra pagine su rami diversi della gerarchia.
% all'inizio è stato fatto il template in php, poi le pagine php/html aggiungendo il css ed infine il js
\label{sub:struttura}
\subsubsection{Header}
\label{ssub:header}
L'header contiene logo, nav, breadcrumb e due link per il login e il signup. Il logo, situato in alto a sinistra, non rimanda a nessuna pagina in quanto il link per la homepage è già contenuto nel menù sotto la voce Home. I due pulsanti per il login e il signup sono rispettivamente "Accedi" e "Registrati" e saranno entrambi presentati solo nel caso in cui l'utente non abbia ancora effettuato l'accesso. Una volta effettuato l'accesso, al posto dei due link ci sarà il nickname dell'utente e il link "esci" che scollega l'utente e porta alla home.
% subsubsection header (end)
\subsubsection{Nav}
\label{ssub:nav}
Il nav del sito contiene il menù e la barra di ricerca.
Il menù occupa la posizione centrale del header e contiene i link alle pagine principali del sito:
\begin{itemize}
	\item Home page;
	\item Pagina relativa ai primi;
	\item Pagina relativa ai secondi;
	\item Pagina relativa ai dolci;
\end{itemize}
La barra di ricerca è situata sotto il menù in posizione centrale ed è ben visibile per permettere agli utenti di cercare le ricette di cui hanno bisogno.
% subsubsection nav (end)
\subsubsection{Breadcrumb}
\label{ssub:breadcrumb}
Il breadcrumb è posizionato a sinistra sotto la barra di ricerca e serve a identificare la posizione dell'utente all'interno del sito. L'ultimo campo corrisponde alla pagina corrente e, per evitare link circolari, è un testo.
Il breadcrumb è presente in tutte le pagine tranne in quella di signup e login, in quanto sono raggiungibili da qualsiasi pagina del sito.
% subsubsection breadcrumb (end)
\subsubsection{Content}
\label{ssub:content}
Lo scopo principale del sito è cercare e leggere le ricette interessate.
\begin{itemize}
	\item Pagina Home: è la prima pagina ad essere visualizzata quando un utente visita il sito ed è presente una breve desrizione per far capire agli utenti cosa possono trovare all'interno del sito.
	Sono poi presenti le tre ricette (una per portata) col voto più alto.

	\item Pagina delle ricette: è la pagina in cui viene presentata la ricetta selezionata. Per prima cosa, è visibile il titolo della ricetta, l'immagine del piatto e sulla destra la lista degli ingredienti, la difficoltà e la durata della ricetta. Sotto è descritto il procedimento della ricetta. Alla fine di ogni ricetta è possibile darle un voto da 1 a 5. Infine c'è la sezione dei commenti in cui è possibile inserire un commento relativo alla ricetta e leggere i commenti degli altri utenti.
	\item Pagina delle portate: la pagina dei primi, secondi e dolci hanno la stessa struttara: in ogni pagina viene presentata la lista di tutte le ricette relative alla portata selezionata. Le ricette sono presentate in riquadri contenenti l'immagine del piatto, il nome, la difficoltà, il tempo, il voto medio e il link "apri" che porta alla pagina della ricetta selezionata.
	\item Pagina Login e Signup: l'utente ha la possibilità di registrarsi o accedere con le proprie credenziali tramite le pagine di login e signup. Nella pagina signup sono presenti i form per inserire la mail, il nickname e la password (sarà necessario inserirla in due form diversi per verificarne la correttezza). Per il login basterà inserire solo il nickname e la password.
	\item Pagina Utente: una volta effettuato l'accesso, l'utente può visitare la propria area personale e verificare le proprie informazioni correnti. C'è la possibilità di eliminare l'account inserendo la propria password nel form apposito in fondo alla pagina.
	\item Pagina Modifica Utente: dalla pagina utente è possibile arrivare alla pagina modifica utente in cui sono presenti tutti i form necessari per cambiare tutte le informazioni dell'utente (immagine, nickname, email e password). Ogni dopo ogni modifica sarà necessario inserire la password per confermare l'operazione. Nel caso si vogliano annullare le modifiche, lo si può fare mediante il tasto annulla, in fondo alla pagina.
	\item Pagina Aggiungi e Modifica ricetta: queste pagine sono raggiungibili solo se si è admin. In entrambe le pagine sono presenti i form per l'inserimento delle informazioni relative alla ricetta che poi compariranno nella pagina della ricetta. Nella pagina "modifica ricetta" i form sono già compilati e sarà possibile apportare modifiche. Nel caso si decidano di annullare le modifiche, è presente il pulsante annulla in fondo alla pagina.
\end{itemize}
% subsubsection content (end)
\subsubsection{Footer}
\label{ssub:footer}
Il footer contiene la dicitura per il copyright, i nomi degli autori del sito, e le icone di certificazione della validità rispetto agli standard W3C di HTML e CSS2.

% subsubsection footer (end)
% subsection struttura (end)
\subsection{Visualizzazione} %le pagine devono essere accessibili indipendentemente dalle dimensioni del dispositivo e del browser
\label{sub:visualizzazione}
\subsubsection{Convenzioni interne}
\label{ssub:convenzioni_interne}
Una delle convenzioni interne al sito riguarda i link: sono sottolineati per permettere l'immediata indivisuazione, e quelli visitati diventano gialli.
% subsubsection convenzioni_interne (end)
\subsubsection{Desktop}
\label{ssub:desktop}

% subsubsection desktop (end)
\subsubsection{Mobile}
\label{ssub:mobile}
Nella visualizzazione da mobile è stato implementato il menù ad hamburger, ovvero il menù è nascosto e per farlo comparire serve premere il burger menù situato in alto a destra. Il pulsante è facilmente raggiungibile per gli utenti che usano il telefono con due mani, mentre gli utenti destri che usano il telefano con una sola mano potranno aprire comodamente il menù col pollice.
Una volta aperto, il menù occupa quasi tutto lo schermo con le voci al centro. Sono presenti inoltre i link per il login e il signup e la barra di ricerca posizionata sul fondo del menù. Si è deciso di inserire tutte le voci insieme alla barra di ricerca nel menù, per avere tutti gli elementi necessari per la ricerca raccolti in un unico spazio. Quando il menù è chiuso è visibile solo il logo affiancato dal tasto del menù.


% subsubsection mobile (end)
\subsubsection{Stampa}
\label{ssub:stampa}
Per quanto riguarda la stampa, sono stati eliminati gli elementi di presentazione (immagini e background) e sono stati tenuti solo gli elemeti necessari, ovvero il contenuto della pagina. È stato rimosso il menù, la barra di ricerca e il footer in quanto non necessari per la visualizzazione della pagina stampata, mentre sono stati tenuti il breadcrumb per specificare il percorso alla pagina selezionata, e il logo per distinguere il sito.
Nella pagina home e quelle delle portate, in cui vengono presentate le varie ricette, è indicato il link che porta ad ogni ricetta.
Le pagine sono in bianco e nero per dare valore al contenuto anzichè alla presentazione.
% subsubsection stampa (end)
% subsection visualizzazione (end)
% section fase_di_progettazione (end)