\section{Fase di progettazione}
\label{sec:fase_di_progettazione}
\subsection{Struttura}
\label{sub:struttura}
\subsubsection{Header}
\label{ssub:header}
L'header contiene logo, nav, breadcrumb e due link per il login e il signup. Il logo, situato in alto a sinistra, non rimanda a nessuna pagina in quanto il link per la homepage è già contenuto nel menu sotto la voce Home. I due pulsanti per il login e il signup sono rispettivamente ``Accedi'' e ``Registrati'' e sono entrambi presenti solo nel caso in cui l'utente non abbia ancora effettuato l'accesso. Una volta effettuato l'accesso, al posto dei due link c'è il nickname dell'utente e il link ``esci'' che scollega l'utente e lo porta alla home.
% subsubsection header (end)
\paragraph{Nav}
\label{par:nav}
Il nav del sito contiene il menu e la barra di ricerca.
Il menu occupa la posizione centrale dell'header e contiene i link alle pagine principali del sito:
\begin{itemize}
	\item Home page;
	\item Pagina relativa ai primi piatti;
	\item Pagina relativa ai secondi piatti;
	\item Pagina relativa ai dolci;
\end{itemize}
La barra di ricerca è situata sotto il menu in posizione centrale ed è ben visibile per permettere agli utenti di cercare le ricette di cui hanno bisogno.
% paragraph nav (end)
\paragraph{Breadcrumb}
\label{par:breadcrumb}
Il breadcrumb è posizionato a sinistra sotto la barra di ricerca e serve a identificare la posizione dell'utente all'interno del sito. L'ultimo campo corrisponde alla pagina corrente e, per evitare link circolari, è un testo.
% paragraph breadcrumb (end)
\subsubsection{Content}
\label{ssub:content}
Lo scopo principale del sito è cercare e consultare le ricette a cui si è interessati.

\paragraph{Pagina Home}
È la prima pagina ad essere visualizzata quando un utente visita il sito, contiene una breve descrizione dei contenuti.
Sono poi presenti le tre ricette (una per portata) con il voto più alto.

\paragraph{Pagine delle portate}
Le pagine con gli elenchi dei primi piatti, i secondi piatti e i dolci hanno la stessa struttura: in ogni pagina viene presentata la lista di tutte le ricette relative alla portata selezionata. Le ricette sono presentate in riquadri contenenti l'immagine del piatto, il nome, la difficoltà, il tempo necessario allo svolgimento, il voto medio e il link ``apri'' che porta alla pagina della ricetta selezionata.

\paragraph{Pagine delle ricette}
È la pagina in cui viene presentata la ricetta selezionata. Per prima cosa, è visibile il titolo della ricetta, l'immagine del piatto e sulla destra la lista degli ingredienti, la difficoltà e la durata della ricetta; in seguito è descritto il procedimento, seguito da un form che permette di assegnare alla ricetta un voto da 1 a 5. Infine c'è la sezione dei commenti in cui è possibile inserire un commento relativo alla ricetta e leggere i commenti degli altri utenti.

\paragraph{Pagine di Login e Signup}
L'utente ha la possibilità di registrarsi o accedere con le proprie credenziali tramite le pagine di login e signup. Nella pagina signup sono presenti dei campi per inserire l'indirizzo email, il nickname e la password (quest'ultima da inserire due volte per confermare la correttezza). Per il login è necessario inserire  il nickname e la password.

\paragraph{Pagina Utente}
Una volta effettuato l'accesso, l'utente può visitare la propria area personale e verificare le proprie informazioni correnti. C'è la possibilità di eliminare l'account inserendo la propria password nel form apposito in fondo alla pagina.

\paragraph{Pagina Modifica Utente}
Dalla pagina utente è possibile aprire la pagina modifica utente in cui sono presenti tutti i form necessari per cambiare le informazioni dell'utente (immagine, nickname, email e password). Ciascun form permette la modifica di un unica informazione, e deve essere confermato inserendo la password attuale. Nel caso si vogliano annullare le modifiche, lo si può fare mediante il pulsante ``Annulla'', in fondo alla pagina.

\paragraph{Pagine Aggiungi e Modifica ricetta}
Queste pagine sono raggiungibili solo dagli utenti con permessi amministratore. In entrambe le pagine sono presenti i form per l'inserimento delle informazioni che poi compariranno nella pagina della ricetta. Nella pagina ``modifica ricetta'' i form sono già compilati con le informazioni attuali. Anche in questo caso, per annullare le modifiche è presente il pulsante ``Annulla'' in fondo alla pagina.

% subsubsection content (end)
\subsubsection{Footer}
\label{ssub:footer}
Il footer contiene la dicitura per il copyright, i nomi degli autori del sito, e le icone di certificazione della validità rispetto agli standard W3C di XHTML e CSS3.

% subsubsection footer (end)
% subsection struttura (end)
\subsection{Visualizzazione} %le pagine devono essere accessibili indipendentemente dalle dimensioni del dispositivo e del browser
\label{sub:visualizzazione}
\subsubsection{Convenzioni interne}
\label{ssub:convenzioni_interne}
Una delle convenzioni interne al sito riguarda i link: sono sottolineati per permettere l'immediata indivisuazione, e quelli visitati diventano gialli.
% subsubsection convenzioni_interne (end)
\subsubsection{Desktop}
\label{ssub:desktop}

% subsubsection desktop (end)
\subsubsection{Mobile}
\label{ssub:mobile}
Nella visualizzazione da mobile è stato implementato il menù ad hamburger, ovvero il menù è nascosto e per farlo comparire serve premere il burger menù situato in alto a destra. Il pulsante è facilmente raggiungibile per gli utenti che usano il telefono con due mani, mentre gli utenti destri che usano il telefano con una sola mano potranno aprire comodamente il menù col pollice.
Una volta aperto, il menù occupa quasi tutto lo schermo con le voci al centro. Sono presenti inoltre i link per il login e il signup e la barra di ricerca posizionata sul fondo del menù. Si è deciso di inserire tutte le voci insieme alla barra di ricerca nel menù, per avere tutti gli elementi necessari per la ricerca raccolti in un unico spazio. Quando il menù è chiuso è visibile solo il logo affiancato dal tasto del menù.


% subsubsection mobile (end)
\subsubsection{Stampa}
\label{ssub:stampa}
Per quanto riguarda la stampa, sono stati eliminati gli elementi di presentazione (immagini e background) e sono stati tenuti solo gli elemeti necessari, ovvero il contenuto della pagina. È stato rimosso il menù, la barra di ricerca e il footer in quanto non necessari per la visualizzazione della pagina stampata, mentre sono stati tenuti il breadcrumb per specificare il percorso alla pagina selezionata, e il logo per distinguere il sito.
Nella pagina home e quelle delle portate, in cui vengono presentate le varie ricette, è indicato il link che porta ad ogni ricetta.
Le pagine sono in bianco e nero per dare valore al contenuto anzichè alla presentazione.
% subsubsection stampa (end)
% subsection visualizzazione (end)
% section fase_di_progettazione (end)
