\section{Fase di testing}\label{sec:fase_di_testing}
La validazione del sito web è una fase necessaria su cui è stata posta particolare attenzione. Tutte le pagine del sito sono state accuratamente validate attraverso molteplici strumenti, di seguito elencati:
\begin{itemize}
    \item Per la validazione del codice HTML è stata utilizzata l'estensione \textbf{Web developer} che fa uso del validatore HTML di W3C. Questa estensione è stata molto utile poiché la maggior parte delle pagine del sito viene costruita dinamicamente.
    \item Per la validazione dei fogli di stile in CSS è stato utilizzato il validatore di W3C, \textbf{CSS Validation Service}.
    \item Per controllare i giusti livelli di contrasto di colore presenti nel sito è stata utilizzata principalmente l'estensione \textbf{Wave - web accessibility evaluation tool}.
    \item Infine è stata utilizzata l'estensione \textbf{Silktide - disability simulator} per simulare alcune delle disabilità comuni.
\end{itemize}
Inoltre il sito risulta funzionante sui browser più comunemente usati, quali Mozilla Firefox, Chrome, Opera e Internet Explorer 11. In quest'ultimo abbiamo notato che la proprietà CSS \csscode{object-fit} relativa alle immagini viene ignorata dal browser, facendo visualizzare così le immagini delle ricette leggermente allungate se non conformi a certe dimensioni. Poiché questo problema non crediamo sia particolarmente rilevante e sopratutto evitabile se le dimensioni delle immagini corrispondono a specifiche dimensioni, il gruppo ha deciso di far conoscere all'utente nelle pagine di caricamento immagini, quali sono le dimensioni preferibili.
% contrasti dei colori AA
% section fase_di_testing (end)