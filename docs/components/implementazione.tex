\section{Fase di implementazione}
\label{sec:fase_di_implementazione}
\subsection{Linguaggi}
\paragraph{HTML}
Tutte le pagine del sito sono state realizzate utilizzando come linguaggio di markup XHTML piuttosto che HTML5, per garantire un certo grado di qualità del codice e alta compatibilità con i browser più obsoleti. In particolare le differenze più importanti che ci hanno portato a preferire XHTML ad HTML5 sono:
\begin{itemize}
    \item I tag <html>, <head>, <title> e <body> sono obbligatori;
    \item Gli elementi devono essere nidificati correttamente;
    \item Gli elementi devono essere sempre chiusi;
    \item Gli elementi devono essere sempre in minuscolo;
    \item I nomi degli attributi devono essere sempre in minuscolo;
\end{itemize}
Per le form di registrazione e accesso, abbiamo scelto di utilizzare come input per le e-mail il tipo \texttt{text}, che grazie ad opportuni controlli specificati nelle sezioni \ref{par:php} e \ref{par:javascript}, si comporta quasi similmente al tipo \texttt{email} di HTML5. % la stessa cosa è stata fatta per il tipo search per la ricerca, number per tempo, range per difficoltà

\paragraph{CSS}

\paragraph{SQL}

\paragraph{PHP}\ref{par:php}
% la struttura principale di ogni pagina è organizzata secondo un template, nello specifico una classe wrapper che fornisce i metodi setter per costruire una pagina. Più in dettaglio, i metodi sono: setBlabla()
% pagine 40* ed eventuali errori
% funzionalità di ricerca
% query portata
Le funzioni di validazione lato server sono contenute all'interno del file \texttt{validation.php}. Sono disponibili:
\begin{itemize}
	\item 
\end{itemize}

\paragraph{Javascript}\ref{par:javascript}
% commenti, voti, validazione
Il linguaggio Javascript è stato utilizzato per implementare il comportamento dinamico lato client di alcune pagine del sito, nello specifico i commenti e i voti nella pagina delle ricette, il menu ad hamburger presente nella visualizzazione mobile.
Poiché è possibile che non tutti gli utenti dispongano della tecnologia adatta, le funzionalità implementate in Javascript sono minime. \newline
I controlli lato client che sono stati implementati sono: 
\begin{itemize}
	\item \texttt{checkEmail(email)} che, passato come parametro la stringa che rappresenta l'id relativo alla mail presente nella form in esame, controlla sia se l'input inserito dall'utente è una stringa vuota, sia se non corrisponde all'espressione regolare che identifica le e-mail.
	\item \texttt{checkNickname(nickname)} che, passato come parametro la stringa che rappresenta l'id relativo al nickname presente nella form in esame, controlla se l'input inserito dall'utente è una stringa vuota e se non corrisponde all'espressione regolare che identifica stringhe contenti solo lettere e numeri di lunghezza tra 3 e 20 caratteri.
	\item \texttt{isPasswordEqual(password, passwordConfirm)} che, passati come parametri le stringhe gli id che identificano l'input relativo alla password e la conferma della password, controlla se le due password sono uguali tra loro.
	\item \texttt{checkImage(image)} che, data la stringa che rappresenta l'id per l'inserimento del file, controlla se l'estensione del file appartiene alle estensioni delle immagini accettate e se la dimensione del file è minore di 150KB.
	\item \texttt{checkTitolo(titolo)} che, passato come parametro la stringa che rappresenta l'id relativo al titolo di una ricetta, controlla se l'input inserito dall'utente è una stringa vuota e se la lunghezza della stringa è tra 3 e 55 caratteri.
	\item \texttt{checkDifficolta(difficolta)} che, passato come parametro la stringa che rappresenta l'id relativo alla difficoltà di una ricetta, controlla se l'input inserito dall'utente è una stringa vuota e se corrisponde ad un numero tra 1 e 5.
	\item \texttt{checkTempo(tempo)} che, passato come parametro la stringa che rappresenta l'id relativo al titolo di una ricetta, controlla se l'input inserito dall'utente è una stringa vuota e se corrisponde ad un intero positivo, escluso lo 0.
	\item \texttt{checkKeywords(keywords)}
\end{itemize}
Nel caso in cui Javascript non sia disponibile o sia disabilitato, la validazione degli input è effettuata lato server grazie alle funzioni PHP sopra descritte.

% section fase_di_implementazione (end)