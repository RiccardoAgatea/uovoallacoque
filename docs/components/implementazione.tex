\section{Fase di implementazione}
\label{sec:fase_di_implementazione}
\subsection{Linguaggi}
\paragraph{HTML}
Tutte le pagine del sito sono state realizzate utilizzando come linguaggio di markup XHTML piuttosto che HTML5, per garantire un certo grado di qualità del codice e alta compatibilità con i browser più obsoleti. In particolare le differenze più importanti che ci hanno portato a preferire XHTML ad HTML5 sono:
\begin{itemize}
    \item I tag <html>, <head>, <title> e <body> sono obbligatori;
    \item Gli elementi devono essere nidificati correttamente;
    \item Gli elementi devono essere sempre chiusi;
    \item Gli elementi devono essere sempre in minuscolo;
    \item I nomi degli attributi devono essere sempre in minuscolo;
\end{itemize}
Per le form di registrazione e accesso, abbiamo scelto di utilizzare come input per le e-mail il tipo \texttt{text}, che grazie ad opportuni controlli specificati nelle sezioni \ref{par:php} e \ref{par:javascript}, si comporta quasi similmente al tipo \texttt{email} di HTML5. % la stessa cosa è stata fatta per il tipo search per la ricerca, number per tempo, range per difficoltà

\paragraph{CSS}

\paragraph{PHP}\ref{par:php}
% la struttura principale di ogni pagina è organizzata secondo un template, nello specifico una classe wrapper che fornisce i metodi setter per costruire una pagina. Più in dettaglio, i metodi sono: setBlabla()
% pagine 40* ed eventuali errori
% validazione
% funzionalità di ricerca
% query portata

\paragraph{Javascript}\ref{par:javascript}
% commenti, voti, validazione

% section fase_di_implementazione (end)