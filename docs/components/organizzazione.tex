\section{Organizzazione interna}
Affinchè ogni membro apprenda ogni aspetto e argomento nella realizzazione di un sito, si è deciso che ognuno debba implementare parti diverse del sito. Questo ha portato allo scambio di idee, opinioni e alla realizzazione di decisioni più pensate e consapevoli. Per contro, il gruppo ha speso più tempo nella propria formazione.\newline
Nello specifico l'analisi e la progettazione del sito è stata pensata in gruppo, mentre l'implementazione del sito è stata suddivisa tra i membri in questo modo:
\begin{itemize}
	\item Agatea Riccardo ha realizzato alcune pagine PHP nello specifico il template, la pagina di home e la funzionalità di ricerca, ha implementato le funzionalità di connessione al database e di autenticazione
	\item Bosinceanu Ecaterina ha realizzato alcune pagine PHP nello specifico la pagina di ricetta, le funzionalità di validazione lato server delle ricette e la pagina di utente e modifica utente. Per quanto riguarda gli altri linguaggi, ha implementato una parte relativa alla presentazione desktop in CSS, ha implementato la validazione lato client delle ricette in Javascript e ha contribuito nella stesura della relazione.
	\item Righetto Sara ha realizzato alcune pagine PHP nello specifico le pagine di registrazione e accedi, la pagina per aggiungere una ricetta, la pagina per la modifica di una ricetta, la funzionalità di eliminazione di una ricetta e le funzionalità di validazione lato server dell'accesso, registrazione e utente. Per quanto riguarda gli altri linguaggi, ha implementato una parte relativa alla presentazione mobile in CSS, ha implementato la validazione lato client dell'accesso, registrazione e utente in Javascript e ha contribuito nella stesura della relazione.
	\item Schiavon Rebecca ha realizzato alcune pagine PHP nello specifico la pagina generica degli elenchi che comprende \textit{primi piatti}, \textit{secondi piatti} e \textit{dolci}, le funzionalità relativa ai piatti più votati e la paginazione degli elenchi. Per quanto riguarda gli altri linguaggi, ha implementato una parte relativa alla presentazione mobile, desktop e stampa in CSS, ha implementato il menu ad hamburger disponibile per mobile in Javascript, ha creato il database e ha contribuito nella stesura della relazione.
\end{itemize}
