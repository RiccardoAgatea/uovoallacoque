\section{Organizzazione interna}
Affinché ogni membro apprenda ogni aspetto e argomento nella realizzazione di un sito, si è deciso che ognuno debba implementare parti diverse del sito. Questo ha portato allo scambio di idee, opinioni e alla realizzazione di decisioni più pensate e consapevoli. Per contro, il gruppo ha speso più tempo nella propria formazione.\newline
Nello specifico l'analisi e la progettazione del sito è stata pensata in gruppo, mentre l'implementazione del sito è stata suddivisa tra i membri in questo modo:
\begin{itemize}
	\item Agatea Riccardo ha creato il database, ha realizzato il template, la homepage e la struttura dei commenti, ed ha implementato la costruzione delle pagine, la connessione al database, l'autenticazione degli utenti, le funzionalità di modifica dei dati personali e le funzionalità di aggiunta, modifica ed eliminazione dei commenti. Per quanto riguarda gli altri linguaggi, ha implementato la presentazione dei commenti in CSS ed ha contribuito alla stesura della relazione.
	\item Bosinceanu Ecaterina ha realizzato la pagina delle ricette, la pagina del profilo utente e la pagina di modifica utente, ed ha implementato le funzionalità di validazione lato server delle ricette. Per quanto riguarda gli altri linguaggi, ha implementato una parte relativa alla presentazione desktop in CSS, ha implementato la validazione lato client delle ricette in Javascript ed ha contribuito alla stesura della relazione.
	\item Righetto Sara ha realizzato le pagine di registrazione e login e le pagine per l'aggiunta e la modifica di una ricetta, ed ha implementato la funzionalità di eliminazione di una ricetta e le funzionalità di validazione lato server dell'accesso e della registrazione utente. Per quanto riguarda gli altri linguaggi, ha implementato una parte relativa alla presentazione mobile in CSS, ha implementato la validazione lato client dell'accesso, registrazione e utente in Javascript ed ha contribuito alla stesura della relazione.
	\item Schiavon Rebecca ha realizzato la pagina degli elenchi (che in base ai parametri dell'URL si presenta come pagina dei \textit{primi piatti}, dei \textit{secondi piatti}, dei \textit{dolci}, o dei \textit{risultati di ricerca}) e le pagine degli errori, ed ha implementato la paginazione degli elenchi e dei commenti, le funzionalità per individuare i piatti più votati e le condizioni per il reindirizzamento alle pagine di errore. Per quanto riguarda gli altri linguaggi, ha implementato una parte relativa alla presentazione mobile, desktop e stampa in CSS, ha implementato il menu ad hamburger disponibile per mobile in Javascript, ed ha contribuito alla stesura della relazione.
\end{itemize}
