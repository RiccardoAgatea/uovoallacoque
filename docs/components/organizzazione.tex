\section{Organizzazione interna}
% ripartizione dei compiti, organizzazione e ore
Affinchè ogni membro apprenda ogni aspetto e argomento nella realizzazione di un sito, si è deciso che ognuno debba implementare parti diverse del sito. Questo ha portato allo scambio di idee, opinioni e alla realizzazione di decisioni più pensate e consapevoli. Per contro, il gruppo ha speso più tempo nella propria formazione. \newline
Nello specifico l'analisi e la progettazione del sito è stata pensata in gruppo, mentre l'implementazione del sito è stata suddivisa tra i membri in questo modo:
\begin{itemize}
	\item Agatea Riccardo ha realizzato alcune pagine PHP nello specifico il template, la pagina di home e la funzionalità di ricerca.
	\item Bosinceanu Ecaterina ha realizzato alcune pagine PHP nello specifico la pagina di ricetta, la pagina di utente e modifica utente, le funzionalità di validazione delle ricette. % css desktop, js validazione delle ricette, stesura della relazione
	\item Righetto Sara ha realizzato alcune pagine PHP nello specifico le pagine di registrazione e accedi, la pagina per aggiungere una ricetta, la pagina per la modifica di una ricetta, la funzionalità di eliminazione di una ricetta e le funzionalità di validazione dell'accesso, registrazione e utente. % css mobile un pochettino, js validazione del resto, stesura della relazione
	\item Schiavon Rebecca ha realizzato alcune pagine PHP nello specifico la pagina generica degli elenchi che comprende \textit{primi piatti}, \textit{secondi piatti} e \textit{dolci}, le funzionalità relativa ai piatti più votati e la paginazione degli elenchi. % css un po tutti
\end{itemize}

% all'inizio è stato fatto il template, poi le pagine php/html aggiungendo il css ed infine il js