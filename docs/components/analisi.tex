\section{Fase di analisi}
\subsection{Analisi delle caratteristiche dell'utenza}
\label{sub:analisi_delle_caratteristiche_dell_utenza}
\subsubsection{Destinatari}
\label{subs:destinatari}
Il sito \emph{uovo alla coque} si rivolge sia ai ragazzi alle prime armi in cucina, che ad adulti che vogliono ampliare le proprie conoscenze culinarie. Essendo rivolto anche a persone con poca dimestichezza nella navigazione in Internet, il sito è particolarmente semplice e intuitivo. Sono state individuate tre tipologie di utenti:
\begin{itemize}
    \item l'\textit{utente generico}, che non è registrato al sito o non ha effettuato il login;
    \item l'\textit{utente registrato}, che ha effettuato il login;
    \item l'\textit{utente amministratore}, che possiede privilegi rispetto agli altri utenti.
\end{itemize}

\subsubsection{Funzionalità}
\label{subs:funzionalità}
Qualsiasi tipo di utente può navigare all'interno del sito e visualizzare tutte le ricette presenti.
Gli utenti che hanno già un'idea chiara su cosa cercare all'interno del sito possono utilizzare la barra di ricerca, visibile nell'header di ogni pagina, che permette di raggiungere qualsiasi ricetta del sito a partire da parole chiave. Per gli utenti che invece vogliono esplorare e scoprire nuove ricette, la home page offre degli spunti mostrando le ricette migliori per ogni categoria, in base alla valutazione degli utenti. \newline
L'utente generico può creare un proprio account tramite la funzionalità di registrazione. L'utente registrato potrà accedere al proprio account ogni volta che lo desidera. Una volta effettuato l'accesso, l'utente potrà:
\begin{itemize}
    \item visualizzare la propria pagina utente;
    \item modificare i propri dati;
    \item dare voti alle ricette;
    \item visualizzare i commenti degli altri utenti;
    \item aggiungere, modificare o eliminare un proprio commento relativo ad una ricetta;
    \item eliminare il proprio account.
\end{itemize}
L'utente amministratore, oltre a possedere tutte le funzionalità degli altri utenti, può aggiungere una nuova ricetta o modificare ed eliminarne una esistente; inoltre, ha la possibilità di rimuovere i commenti degli altri utenti nel caso i contenuti non rispettino i termini di servizio del sito.

\subsubsection{Accessibilità}
\label{subs:accessibilità}
% fare riferimento al w3c https://www.w3.org/standards/webdesign/accessibility
% subsubsection accessibilità (end)
% subsection analisi_delle_caratteristiche_dell_utenza (end)
