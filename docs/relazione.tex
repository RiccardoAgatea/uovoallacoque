\input{config.tex}
\title{Progetto di Tecnologie Web}
\author{}
\date{Anno accademico 2019/2020}

\begin{document}
	\pagenumbering{gobble}
	\maketitle
	\begin{figure}[H]
		\centering
		\includegraphics[width=8cm]{img/logo.png}
	\end{figure}
	\begin{table}[H]
		\centering
		\begin{tabular}{c|c c c}
			\textbf{Componenti}&Agatea&Riccardo&1170718\\
			&Bosinceanu&Ecaterina&1169669\\
			&Righetto&Sara&1174009\\
			&Schiavon&Rebecca&1163774\\
		\end{tabular}
	\end{table}

	\begin{center}
		\textbf{Link repository}: https://github.com/RiccardoAgatea/uovoallacoque\\
		\textbf{Indirizzo sito web}: http://tecweb1920.studenti.math.unipd.it/ragatea\\
		\textbf{Email referente del gruppo}: riccardo.agatea@studenti.unipd.it
	\end{center}

	\begin{table}[H]
		\centering
		\begin{tabular}{c|c c}
			\textbf{Utenti} & \textbf{E-mail} & \textbf{Password} \\
			\hline
			Admin & admin@gmail.com & admin \\
			User & user@gmail.com & user \\
		\end{tabular}
	\end{table}
	\newpage
	\pagenumbering{roman}
	\tableofcontents
	\newpage
	\pagenumbering{arabic}
	\renewcommand{\abstractname}{Abstract}
	\begin{abstract}
		Il sito \emph{Uovo alla coque} è un sito di ricette a tema uovo, ovvero tutte le ricette al suo interno hanno come ingrediente principale o necessario l'uovo. Il nome del sito rispecchia questa caratteristica perchè rimanda alla ricetta più famosa con cui è possibile cucinare un uovo. \newline
		Il sito permette a tutti gli utenti di visualizzare tutte le ricette del sito, divise per portata (primi piatti, secondi piatti e dolci), e cercare le ricette che più si desiderano. Inoltre l'utente registrato può commentare e votare le ricette preferite, mentre l'utente amministratore può anche aggiungere, modificare ed eliminare le ricette.
	\end{abstract}
	\newpage
	\section{Fase di analisi}
\subsection{Analisi delle caratteristiche dell'utenza}
\label{sub:analisi_delle_caratteristiche_dell_utenza}
\subsubsection{Destinatari}
\label{subs:destinatari}
Il sito \emph{uovo alla coque} si rivolge sia ai ragazzi alle prime armi in cucina, che ad adulti che vogliono ampliare le proprie conoscenze culinarie. Essendo rivolto anche a persone con poca dimestichezza nella navigazione in Internet, il sito è particolarmente semplice e intuitivo. Sono state individuate tre tipologie di utenti:
\begin{itemize}
    \item l'\textit{utente generico}, che non è registrato al sito o non ha effettuato il login;
    \item l'\textit{utente registrato}, che ha effettuato il login;
    \item l'\textit{utente amministratore}, che possiede privilegi rispetto agli altri utenti.
\end{itemize}

\subsubsection{Funzionalità}
\label{subs:funzionalità}
Qualsiasi tipo di utente può navigare all'interno del sito e visualizzare tutte le ricette presenti.
Gli utenti che hanno già un'idea chiara su cosa cercare all'interno del sito possono utilizzare la barra di ricerca, visibile nell'header di ogni pagina, che permette di raggiungere qualsiasi ricetta del sito a partire da parole chiave. Per gli utenti che invece vogliono esplorare e scoprire nuove ricette, la home page offre degli spunti mostrando le ricette migliori per ogni categoria, in base alla valutazione degli utenti. \newline
L'utente generico può creare un proprio account tramite la funzionalità di registrazione. L'utente registrato potrà accedere al proprio account ogni volta che lo desidera. Una volta effettuato l'accesso, l'utente potrà:
\begin{itemize}
    \item visualizzare la propria pagina utente;
    \item modificare i propri dati;
    \item dare voti alle ricette;
    \item visualizzare i commenti degli altri utenti;
    \item aggiungere, modificare o eliminare un proprio commento relativo ad una ricetta;
    \item eliminare il proprio account.
\end{itemize}
L'utente amministratore, oltre a possedere tutte le funzionalità degli altri utenti, può aggiungere una nuova ricetta o modificare ed eliminarne una esistente; inoltre, ha la possibilità di rimuovere i commenti degli altri utenti nel caso i contenuti non rispettino i termini di servizio del sito.

\subsubsection{Accessibilità}
\label{subs:accessibilità}
% fare riferimento al w3c https://www.w3.org/standards/webdesign/accessibility
% subsubsection accessibilità (end)
% subsection analisi_delle_caratteristiche_dell_utenza (end)

	\newpage
	\section{Fase di progettazione}
\label{sec:fase_di_progettazione}
\subsection{Struttura}
\label{sub:struttura}
\subsubsection{Header}
\label{ssub:header}
L'header contiene logo, nav, breadcrumb e due link per il login e il signup. Il logo, situato in alto a sinistra, non rimanda a nessuna pagina in quanto il link per la homepage è già contenuto nel menu sotto la voce Home. I due pulsanti per il login e il signup sono rispettivamente ``Accedi'' e ``Registrati'' e sono entrambi presenti solo nel caso in cui l'utente non abbia ancora effettuato l'accesso. Una volta effettuato l'accesso, al posto dei due link c'è il nickname dell'utente e il link ``esci'' che scollega l'utente e lo porta alla home.
% subsubsection header (end)
\paragraph{Nav}
\label{par:nav}
Il nav del sito contiene il menu e la barra di ricerca.
Il menu occupa la posizione centrale dell'header e contiene i link alle pagine principali del sito:
\begin{itemize}
	\item Home page;
	\item Pagina relativa ai primi piatti;
	\item Pagina relativa ai secondi piatti;
	\item Pagina relativa ai dolci;
\end{itemize}
La barra di ricerca è situata sotto il menu in posizione centrale ed è ben visibile per permettere agli utenti di cercare le ricette di cui hanno bisogno.
% paragraph nav (end)
\paragraph{Breadcrumb}
\label{par:breadcrumb}
Il breadcrumb è posizionato a sinistra sotto la barra di ricerca e serve a identificare la posizione dell'utente all'interno del sito. L'ultimo campo corrisponde alla pagina corrente e, per evitare link circolari, è un testo.
% paragraph breadcrumb (end)
\subsubsection{Content}
\label{ssub:content}
Lo scopo principale del sito è cercare e consultare le ricette a cui si è interessati.

\paragraph{Pagina Home}
È la prima pagina ad essere visualizzata quando un utente visita il sito, contiene una breve descrizione dei contenuti.
Sono poi presenti le tre ricette (una per portata) con il voto più alto.

\paragraph{Pagine delle portate}
Le pagine con gli elenchi dei primi piatti, i secondi piatti e i dolci hanno la stessa struttura: in ogni pagina viene presentata la lista di tutte le ricette relative alla portata selezionata. Le ricette sono presentate in riquadri contenenti l'immagine del piatto, il nome, la difficoltà, il tempo necessario allo svolgimento, il voto medio e il link ``apri'' che porta alla pagina della ricetta selezionata.

\paragraph{Pagine delle ricette}
È la pagina in cui viene presentata la ricetta selezionata. Per prima cosa, è visibile il titolo della ricetta, l'immagine del piatto e sulla destra la lista degli ingredienti, la difficoltà e la durata della ricetta; in seguito è descritto il procedimento, seguito da un form che permette di assegnare alla ricetta un voto da 1 a 5. Infine c'è la sezione dei commenti in cui è possibile inserire un commento relativo alla ricetta e leggere i commenti degli altri utenti.

\paragraph{Pagine di Login e Signup}
L'utente ha la possibilità di registrarsi o accedere con le proprie credenziali tramite le pagine di login e signup. Nella pagina signup sono presenti dei campi per inserire l'indirizzo email, il nickname e la password (quest'ultima da inserire due volte per confermare la correttezza). Per il login è necessario inserire  il nickname e la password.

\paragraph{Pagina Utente}
Una volta effettuato l'accesso, l'utente può visitare la propria area personale e verificare le proprie informazioni correnti. C'è la possibilità di eliminare l'account inserendo la propria password nel form apposito in fondo alla pagina.

\paragraph{Pagina Modifica Utente}
Dalla pagina utente è possibile aprire la pagina modifica utente in cui sono presenti tutti i form necessari per cambiare le informazioni dell'utente (immagine, nickname, email e password). Ciascun form permette la modifica di un unica informazione, e deve essere confermato inserendo la password attuale. Nel caso si vogliano annullare le modifiche, lo si può fare mediante il pulsante ``Annulla'', in fondo alla pagina.

\paragraph{Pagine Aggiungi e Modifica ricetta}
Queste pagine sono raggiungibili solo dagli utenti con permessi amministratore. In entrambe le pagine sono presenti i form per l'inserimento delle informazioni che poi compariranno nella pagina della ricetta. Nella pagina ``modifica ricetta'' i form sono già compilati con le informazioni attuali. Anche in questo caso, per annullare le modifiche è presente il pulsante ``Annulla'' in fondo alla pagina.

% subsubsection content (end)
\subsubsection{Footer}
\label{ssub:footer}
Il footer contiene la dicitura per il copyright, i nomi degli autori del sito, e le icone di certificazione della validità rispetto agli standard W3C di XHTML e CSS3.

% subsubsection footer (end)
% subsection struttura (end)
\subsection{Visualizzazione} %le pagine devono essere accessibili indipendentemente dalle dimensioni del dispositivo e del browser
\label{sub:visualizzazione}

\subsubsection{Desktop}
\label{ssub:desktop}

% subsubsection desktop (end)
\subsubsection{Mobile}
\label{ssub:mobile}
Nella visualizzazione da mobile è stato implementato il menù ad hamburger, ovvero il menù è nascosto e per farlo comparire serve premere il burger menù situato in alto a destra. Il pulsante è facilmente raggiungibile per gli utenti che usano il telefono con due mani, mentre gli utenti destri che usano il telefano con una sola mano potranno aprire comodamente il menù col pollice.
Una volta aperto, il menù occupa quasi tutto lo schermo con le voci al centro. Sono presenti inoltre i link per il login e il signup e la barra di ricerca posizionata sul fondo del menù. Si è deciso di inserire tutte le voci insieme alla barra di ricerca nel menù, per avere tutti gli elementi necessari per la ricerca raccolti in un unico spazio. Quando il menù è chiuso è visibile solo il logo affiancato dal tasto del menù.


% subsubsection mobile (end)
\subsubsection{Stampa}
\label{ssub:stampa}
Per quanto riguarda la stampa, sono stati eliminati gli elementi di presentazione (immagini e background) e sono stati tenuti solo gli elemeti necessari, ovvero il contenuto della pagina. È stato rimosso il menù, la barra di ricerca e il footer in quanto non necessari per la visualizzazione della pagina stampata, mentre sono stati tenuti il breadcrumb per specificare il percorso alla pagina selezionata, e il logo per distinguere il sito.
Le pagine sono in bianco e nero per dare valore al contenuto anzichè alla presentazione.
% subsubsection stampa (end)
% subsection visualizzazione (end)

\subsection{Accessibilità}
\label{sub:accessibilità}
\subsubsection{Convenzioni interne}
\label{ssub:convenzioni_interne}
Una delle convenzioni interne al sito riguarda i link: sono sottolineati per permettere l'immediata indivisuazione, e quelli visitati diventano gialli.
% subsubsection convenzioni_interne (end)
% fare riferimento al w3c https://www.w3.org/standards/webdesign/accessibility
% subsection accessibilità (end)

% section fase_di_progettazione (end)

	\newpage
	\section{Fase di implementazione}
\label{sec:fase_di_implementazione}
\subsection{HTML}
\subsection{CSS}
\subsection{PHP}
\subsection{Javascript}

% section fase_di_implementazione (end)
	\newpage
	\section{Fase di testing}\label{sec:fase_di_testing}
La validazione del sito web è una fase necessaria su cui è stata posta particolare attenzione. Tutte le pagine del sito sono state accuratamente validate attraverso molteplici strumenti, di seguito elencati:
\begin{itemize}
    \item Per la validazione del codice HTML è stata utilizzata l'estensione \textbf{Web developer} che fa uso del validatore HTML di W3C. Questa estensione è stata molto utile poiché la maggior parte delle pagine del sito viene costruita dinamicamente.
    \item Per la validazione dei fogli di stile in CSS è stato utilizzato il validatore di W3C, \textbf{CSS Validation Service}.
    \item Per controllare i giusti livelli di contrasto di colore presenti nel sito è stata utilizzata principalmente l'estenzione \textbf{Wave - web accessibility evaluation tool}.
    \item Infine è stata utilizzata l'estensione \textbf{Silktide - disability simulator} per simulare alcune delle disabilità comuni.
\end{itemize}
Inoltre il sito risulta funzionante sui browser più comunemente usati, quali Mozilla Firefox, Chrome, Opera e Internet Explorer 11. 
% contrasti dei colori AA
% section fase_di_testing (end)
	\newpage
	\section{Organizzazione interna}
Affinchè ogni membro apprenda ogni aspetto e argomento nella realizzazione di un sito, si è deciso che ognuno debba implementare parti diverse del sito. Questo ha portato allo scambio di idee, opinioni e alla realizzazione di decisioni più pensate e consapevoli. Per contro, il gruppo ha speso più tempo nella propria formazione.\newline
Nello specifico l'analisi e la progettazione del sito è stata pensata in gruppo, mentre l'implementazione del sito è stata suddivisa tra i membri in questo modo:
\begin{itemize}
	\item Agatea Riccardo ha realizzato alcune pagine PHP nello specifico il template, la pagina di home e la funzionalità di ricerca, ha implementato le funzionalità di connessione al database e di autenticazione
	\item Bosinceanu Ecaterina ha realizzato alcune pagine PHP nello specifico la pagina di ricetta, le funzionalità di validazione lato server delle ricette e la pagina di utente e modifica utente. Per quanto riguarda gli altri linguaggi, ha implementato una parte relativa alla presentazione desktop in CSS, ha implementato la validazione lato client delle ricette in Javascript e ha contribuito nella stesura della relazione.
	\item Righetto Sara ha realizzato alcune pagine PHP nello specifico le pagine di registrazione e accedi, la pagina per aggiungere una ricetta, la pagina per la modifica di una ricetta, la funzionalità di eliminazione di una ricetta e le funzionalità di validazione lato server dell'accesso, registrazione e utente. Per quanto riguarda gli altri linguaggi, ha implementato una parte relativa alla presentazione mobile in CSS, ha implementato la validazione lato client dell'accesso, registrazione e utente in Javascript e ha contribuito nella stesura della relazione.
	\item Schiavon Rebecca ha realizzato alcune pagine PHP nello specifico la pagina generica degli elenchi che comprende \textit{primi piatti}, \textit{secondi piatti} e \textit{dolci}, le funzionalità relativa ai piatti più votati e la paginazione degli elenchi. Per quanto riguarda gli altri linguaggi, ha implementato una parte relativa alla presentazione mobile, desktop e stampa in CSS, ha implementato il menu ad hamburger disponibile per mobile in Javascript, ha creato il database e ha contribuito nella stesura della relazione.
\end{itemize}

\end{document}